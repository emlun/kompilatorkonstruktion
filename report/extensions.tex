A possible extension to the KOOL templates is to implement value parameters for
templates in a similar way to what is possible in C++, meaning that you instead
of only allowing identifiers as template arguments when you define a template
class or method, you also allow primitive types. For example
\textbf{Foo$\langle$Int$\rangle$}, witch means the class should be instantiated
as \textbf{bar = new Foo$\langle$17$\rangle$}. This can be a powerful tool if
you for instance wants to define a matrix class for n-dimensional matrices.

Another interesting extension if you allow value parameters and multiple
class declarations for templates is using first matching templates. Making
this kind of code viable:

\begin{lstlisting}

class Foo<Int>{...}
class Foo<T>{...}
class Foo<T,V>{...}

\end{lstlisting}

And calling it with \textbf{bar = new Foo$\langle$Int$\rangle$} would invoke
the first, \textbf{bar = new Foo$\langle$String$\rangle$} the second and so on.
