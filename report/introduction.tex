We want to implement templates in a similar way to C++, in the appendix you can
see the changes we want to do to the language.

To implement this we will add a new phase between the name analysis and type
checking to expand the templates with the needed types. If we for example define
a method \textbf{max$\langle$T$\rangle$(a:T,b:T):T} and call it with a pair of
Ints and a pair of Strings we need to expand the code into one method for ints
and one for strings, for example \textbf{max\$Int(a:Int,b:Int):Int}.

This in addition to adding a new Type or Symbol for templates is all the
changes we think are necessary for implementing templates. In particular, the
type checking and code generation phases should not need to be modified since
we will require explicit type arguments in references to template types and
methods. Automatic inference of type parameters where possible is proposed as a
further extension if we get the time.

In the name analysis phase we simply add a template symbol to all references to
templates, and in the phase before type checking we expand the template by adding
any needed classes and methods to the AST. The template symbol contains a list of
either methodSymbols or classSymbols for all the versions that has been created.

\iffalse
The first thing our compiled does is checking whether the code compiles to
the BNF, this is done in two steps by first tokenizing the code into the
tokens found in our language, and then in the parser check whether the order
of the tokens are in complient with our language. We had to add some rules to
the BNF to make room for templates, the addition can be found in the appendix.
In the next part, name analysis we make sure that the code follows a set of rules
and attatch symbols to all identifiers. After the name analysis we run type
checking where we detect and make sure that all expressions produce the correct
type. To make this part work with templates we have to expand all templates
befor we run the type checking, and modify the symbols slighetly.
\fi
