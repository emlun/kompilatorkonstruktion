\emph{Give code examples where your extension is useful, and describe how they work
with it. Make sure you include examples where the most intricate features of
your extension are used, so that we have an immediate understanding of what the
challenges are.}

One application where templates are very useful is for general-purpose library
code. The following example illustrates how they can be leveraged for a generic
list class and a sad imitation of Scala's \textbf{Option} monad.

The usefulness of this example comes especially from the fact that the KOOL
grammar doesn't provide a way to check for null values. For example, the Maybe
template below saves us from having to add extra boolean flags all over the place
just to keep track of whether other fields contain values or not --- now we can
encapsulate that logic in a reusable component.

\begin{lstlisting}
// Template class
class Maybe<T> {
  var is: Bool; var value: T;
  def set(newValue: T): Maybe<T> = {
    value = newValue;
    is = true;
    return this;
  }
  def unset(): Maybe<T> = {
    is = false;
    return this;
  }
  def isDefined(): Bool = { return is; }
  def get(): T = { return value; }
}

// Multiple template parameters
class Pair<A, B> {
  var left: A; var right: B;
  def getLeft(): A = { return left; }
  def getRight(): B = { return right; }
  def setLeft(newLeft: A): Pair<A, B> = {
    left = newLeft;
    return this;
  }
  def setRight(newRight: A): Pair<A, B> = {
    right = newRight;
    return this;
  }
}

// Template methods
class MaybeUtil {
  def yep<T>(newValue: T): Maybe<T> = {
    return new Maybe<T>().set(newValue);
  }
  def nope<T>(): Maybe<T> = {
    return new Maybe<T>().unset();
  }
}

// Recursive template class
class LinkedList<T> {
  var head: Maybe<T>; var tail: Maybe<LinkedList<T>>;
  def init(): LinkedList<T> = {
    // Explicit type parameter required
    head = new MaybeUtil().nope<Int>();
    tail = new MaybeUtil().nope<LinkedList<T>>();
    return this;
  }
  def setHead(newHead: T): LinkedList<T> = {
    // Type parameter can be inferred from argument
    head = new MaybeUtil().yep(newHead);
    return this;
  }
  def setTail(newTail: LinkedList<T>): LinkedList<T> = {
    tail = new MaybeUtil().yep(newTail);
    return this;
  }
  def getHead(): Maybe<T> = { return head; }
  def getTail(): Maybe<LinkedList<T>> = { return tail; }
  def prepend(newHead: T): LinkedList<T> = {
    return new LinkedList<T>().init()
      .setHead(newHead).setTail(this);
  }
  def append(item: T): LinkedList<T> = {
    var currentTail: LinkedList<T>;
    var dummy: LinkedList<T>;
    currentTail = this;
    while(currentTail.getTail().isDefined()) {
      currentTail = currentTail.getTail().get();
    }
    dummy = currentTail.setTail(
      new LinkedList<T>().init().setHead(item));
    return this;
  }
  def size(): Int = {
    var result: Int;
    result = 0;
    if(head.isDefined()) {
      result = result + 1;
    }
    if(tail.isDefined()) {
      result = result + tail.get().size();
    }
    return result;
  }
}
\end{lstlisting}

This section should convince us that you understand how your extension can be
useful and that you thought about the corner cases.

There are two main corner cases we need to consider. The first is the optional
type parameter in calls to template methods. If the method doesn't reference the
template type in any way in its argument list then we cannot expand the template,
as illustrated by the following example.

\begin{lstlisting}
class True {
  def get(): Bool = { return true; }
}
class False {
  def get(): Bool = { return false; }
}
class X {
  def make<T>(): T = { return new T(); }
  def identity<T>(a: T): T = { return a; }
}
object Main {
  def main(): Unit = {
    if(new X().identity(new True()).get()) {
      if(new X().make().get()) {
        println("False");
      } else {
        println("True");
      }
      println("Impossible");
    }
  }
}
\end{lstlisting}

Will the program above output ``True'' or ``False''? We can infer the type
parameter to the \textbf{identity} call, but not to the \textbf{make} call.
Therefore, type parameters in method calls are optional if they can be inferred
from the arguments but mandatory if they cannot.

The other main corner case is nested template references, such as
\textbf{new
Maybe$\langle$LinkedList$\langle$LinkedList$\langle$Int$\rangle\rangle\rangle$()}.
We can either run the template expansion step multiple times, or make sure to
always expand deeper levels of nested template references first.
