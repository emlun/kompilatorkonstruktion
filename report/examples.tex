One application where templates are very useful is for general-purpose library
code. The following example illustrates how they can be leveraged for a generic
list class and a sad imitation of Scala's \textbf{Option} monad.

The usefulness of this example comes especially from the fact that the KOOL
grammar doesn't allow arrays of other types than \textbf{Int}, and doesn't
provide a way to check for null values. For example, the Maybe template below
saves us from having to add extra boolean flags all over the place just to keep
track of whether other fields contain values or not --- now we can encapsulate
that logic in a reusable component.

\begin{lstlisting}
// Template class
class Maybe<T> {
  var is: Bool; var value: T;
  def set(newValue: T): Maybe<T> = {
    value = newValue;
    is = true;
    return this;
  }
  def unset(): Maybe<T> = {
    is = false;
    return this;
  }
  def isDefined(): Bool = { return is; }
  def get(): T = { return value; }
}

// Multiple template parameters
class Pair<A, B> {
  var left: A; var right: B;
  def getLeft(): A = { return left; }
  def getRight(): B = { return right; }
  def setLeft(newLeft: A): Pair<A, B> = {
    left = newLeft;
    return this;
  }
  def setRight(newRight: A): Pair<A, B> = {
    right = newRight;
    return this;
  }
}

// Template methods
class MaybeUtil {
  def yep<T>(newValue: T): Maybe<T> = {
    return new Maybe<T>().set(newValue);
  }
  def nope<T>(): Maybe<T> = {
    return new Maybe<T>().unset();
  }
}

// Recursive template class
class LinkedList<T> {
  var head: Maybe<T>; var tail: Maybe<LinkedList<T>>;
  def init(): LinkedList<T> = {
    head = new MaybeUtil().nope<Int>();
    tail = new MaybeUtil().nope<LinkedList<T>>();
    return this;
  }
  def setHead(newHead: T): LinkedList<T> = {
    head = new MaybeUtil().yep<T>(newHead);
    return this;
  }
  def setTail(newTail: LinkedList<T>): LinkedList<T> = {
    tail = new MaybeUtil().yep<T>(newTail);
    return this;
  }
  def getHead(): Maybe<T> = { return head; }
  def getTail(): Maybe<LinkedList<T>> = { return tail; }
  def prepend(newHead: T): LinkedList<T> = {
    return new LinkedList<T>().init()
      .setHead(newHead).setTail(this);
  }
  def append(item: T): LinkedList<T> = {
    var currentTail: LinkedList<T>;
    var dummy: LinkedList<T>;
    currentTail = this;
    while(currentTail.getTail().isDefined()) {
      currentTail = currentTail.getTail().get();
    }
    dummy = currentTail.setTail(
      new LinkedList<T>().init().setHead(item));
    return this;
  }
  def size(): Int = {
    var result: Int;
    result = 0;
    if(head.isDefined()) {
      result = result + 1;
    }
    if(tail.isDefined()) {
      result = result + tail.get().size();
    }
    return result;
  }
}
\end{lstlisting}

The main corner case is nested template references, such as \textbf{new
Maybe\tmpl{LinkedList\tmpl{LinkedList\tmpl{Int}}}()}.
We can either run the template expansion step multiple times, or make sure to
always expand deeper levels of nested template references first.


\subsection{Simple example with expanded code}
This is a simple example showing how templets are expanded into
'real' classes. As we can see in the first section we have a template class
Container\tmpl{T} that in the second part has been expanded into
two 'real' classes, Container\$Int and Container\$String, with coresponding
methods.

\begin{lstlisting}
  object Main {
    def main(): Unit = {
      if(new RealMain().realMain()){}
    }
  }

  class RealMain {
    var intC: Container<Int>;
    var stringC: Container<String>;
    def realMain(): Bool = {
      intC = new Container<Int>().set(7);
      stringC = new Container<String>().set("Number: ");
      println(stringC.get() + intC.get());
      return false;
    }
  }

  class Container<T>{
    var item: T;
    def set(x:T): Container<T> = {
      item = x;
      return this;
    }
    def get(): T = {
      return item;
    }
  }

\end{lstlisting}

The kool template code (or .kpp code for short) above results in the
kool code below.

\begin{lstlisting}
  object Main {
    def main() : Unit = {
      if ( new RealMain().realMain() ){}
    }
  }

  class RealMain {
    var intC : Container$Int;
    var stringC : Container$String;
    def realMain (  ) : Bool = {
      intC = new Container$Int().set ( 7 );
      stringC = new Container$String().set ( "Number" );
      println( (stringC.get (  ) + intC.get (  )) );
      return false;
    }
  }

  class Container$String {
    var item : String;
    def set ( x : String ) : Container$String = {
      item = x;
      return this;
    }

    def get (  ) : String = {
      return item;
    }
  }

  class Container$Int {
    var item : Int;
    def set ( x : Int ) : Container$Int = {
      item = x;
      return this;
    }

    def get (  ) : Int = {
      return item;
    }
  }

\end{lstlisting}
